\chapter{Introduction} \label{ch:Introduction}
The experiments in elementary particle physics probe interactions between sub-atomic particles to probe the fundamental laws of nature. The standard model of particle physics is a well-tested theory that lays out the important parameters for the interaction between particles. Prediction of the existence of the top quark (1995), neutrino (2000), and Higgs (2012) have been confirmed experimentally. This model attempts to describe the electromagnetic, weak, and strong nuclear interactions among subatomic particles.% which are among the four fundamental interactions of nature.

Despite its many successes, the standard model is not a complete theory. For example, gravitational interactions, the matter-antimatter asymmetry in the universe among others, is not predicted in agreement with observations by the current model. There are extensions to the SM that predict the existence of new generations of particles and forces at higher energies accessible with accelerators until recently. 

Some of these ideas can be tested experimentally at the Large Hadron Collider (LHC) at CERN that collides protons at unprecedented high beam energy and intensities to produce particles of higher masses and measure rare particle reactions. At any time the precise knowledge of the rate at which proton-proton collision produce any particles, the so-called luminosity, is crucial to obtain absolute production rate of the new signals that are compared to theoretical predictions and to accurately predict the production of fake signals that need to be subtracted.
The luminosity eventually sets a limit on the confidence of our physics findings. My thesis work concerns systematic studies of the Pixel Luminosity Telescope (PLT) and the determination of background to the luminosity measurement.

This thesis documents corrections applied to the measurement of the luminosity delivered to the CMS's PLT detector at the LHC in proton-proton (pp) collisions at a center-of-mass energy $13$ TeV during the 2015-2016 run period. This thesis is structured the following: Chapter \ref{ch:physBKGD} provides a brief overview of the standard model of particle physics and the concept of luminosity. Chapter \ref{ch:expSetup} describes the experimental setup for the PLT and the method used for calibration.

CMS installed the Pixel Luminosity Telescope (PLT) in 2015; software for analyzing the machine conditions and measurement data had to be built. Chapter \ref{ch:operations} provides an overview of operational work done as part of the thesis. Charged particle reconstruction, likelihood fits to extract signal and background contributions and corrections to the published luminosity values are discussed in chapter \ref{ch:reconstruction} and \ref{ch:Correction}. 
