\chapter{Introduction} \label{ch:Introduction}

The quest for understanding the fundamental nature of the physical world around us has taken us from studying tangible objects in our proximity to studying celestial bodies millions of light years away. The nature of physical world, either at large or small scale, is governed by a common set of laws that we seek to understand. To that end, the experiments in elementary particle physics probe interactions between sub atomic particles to understand the fundamental laws of nature. Standard Model of Particle Physics, originally purposed back in 70s, is a well tested theory that lays out the important parameters for the interaction between particles. Correct prediction for the existence of top quark (1995), neutrino (2000), and Higgs (2012). This model attempts to describe the electromagnetic, weak, and strong nuclear interactions among subatomic particles which are among the four fundamental interactions of nature.

Despite it's many successes, Standard Model (SM), is not a complete theory. Gravitational interactions, matter-antimatter asymmetry in the universe, among others, cannot be explained by the current understanding of the SM. There are extensions to the SM that predict the existence of new generations of particles and forces. Some of these ideas can be tested experimentally at the Large Hadron Collider (LHC) at CERN where protons are collided together to generate and study the interactions of constituent sub-atomic particles. Major part of the experimental effort, therefore, is in increasing the number of proton-proton collisions to get ever more observations for further study and analysis. My thesis work was connected to the systematic studies of the Pixel Luminosity Telescope (PLT) which is one of the dedicated luminometers for the Compact Muon Solenoid (CMS) experiment on the LHC at CERN.

 
An accurate measurement of observed luminosity is one of the important parameter in particle physics experiments. This is because the occurrence of any rare physics events is directly proportional to the amount of collision events $\mathcal{L}$ measured by the detector. As such, luminosity parameter eventually sets limit on the confidence of our physics findings which is why it is equally important to know the backgrounds and uncertainties with our measurement. 

This thesis describes the measurement of the luminosity delivered to the CMS's PLT detector at the LHC in pp collisions at a centre-of-mass energy of $\sqrt{14}= 14 TeV$ during 2015-2016 period. This thesis is structured the following way. Chapter \ref{ch:physBKGD} provides a brief overview of the Standard Model of particle physics and the concept of Luminosity, $mathcal{L}$. Chapter\ref{ch:expSetup} goes into detail about the experimental setup for the Pixel Luminosity Detector, and the VdM scan method used for data calibration. 

PLT came online only in 2015; some systems for analyzing the machine conditions had to be built. Chapter \ref{ch:operations} provides the operational work done as part of the thesis. Finally, Chapter \ref{ch:reconstruction} and \ref{ch:Correction} deal with the track reconstruction, likelihood fits, and correction. Parameters taken from maximum likelyhood fits to the low-luminosity vdm scan are used as a reference to provide corrections to the luminosity at much higher intensity.

