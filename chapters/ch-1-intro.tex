\chapter{Introduction} \label{ch:Introduction}
% keep intro to thesis only?
%The quest for understanding the fundamental nature of the physical world around us has taken us from studying tangible objects in our proximity to studying celestial bodies millions of light years away. The nature of physical world, either at large or small scale, is governed by a common set of laws that we seek to understand. To that end, 
The experiments in elementary particle physics probe interactions between sub atomic particles to probe the fundamental laws of nature. The standard model [ref] of particle physics is a well tested theory that lays out the important parameters for the interaction between particles. Prediction of the existence of the top quark (1995), neutrino (2000), and Higgs (2012) have been confirmed experimentally. This model attempts to describe the electromagnetic, weak, and strong nuclear interactions among subatomic particles.% which are among the four fundamental interactions of nature.

Despite it's many successes, standard model is not a complete theory. For example, gravitational interactions, the matter-antimatter asymmetry in the universe among others, is not predicted in agreement with observations by the current model. There are extensions to the SM that predict the existence of new generations of particles and forces at higher energies accessible with accelerators until recently. 

Some of these ideas can be tested experimentally at the Large Hadron Collider (LHC) at CERN that collides protons at unprecedented high beam energy and intensities to produce particles of higher masses and measure rare particle reactions. At any time the precise knowledge of the rate at which proton proton collision produce any particles, the so called luminosity, is crucial to obtain absolute production rate of the new signals that are compared to theoretical predictions, and to accurately predict the production of fake signals that need to be subtracted.
The luminosity eventually sets a limit on the confidence of our physics findings. My thesis work concerns systematic studies of the Pixel Luminosity Telescope (PLT) and the determination of background to the luminosity measurement.

This thesis documents corrections applied to measurement of the luminosity delivered to the CMS's PLT detector at the LHC in proton proton (pp) collisions at a center-of-mass energy $13$ TeV during the 2015-2016 run period. This thesis is structured the following: Chapter \ref{ch:physBKGD} provides a brief overview of the standard model of particle physics and the concept of luminosity. Chapter \ref{ch:expSetup} describes the experimental setup for the PLT and the method used for calibration.

CMS installed the Pixel Luminosity Telescope (PLT) in 2015; software for analyzing the machine conditions and measurement data had to be built. Chapter \ref{ch:operations} provides an overview of operational work done as part of the thesis. Charged particle reconstruction, likelihood fits to extract signal and background contributions and corrections to the published luminosity values are discussed in chapter \ref{ch:reconstruction} and \ref{ch:Correction}. 

%are collided together to generate and study the interactions of constituent sub-atomic particles. Major part of the experimental effort, therefore, is in increasing the number of proton-proton collisions to get ever more observations for further study and analysis. My thesis work was connected to the systematic studies of the Pixel Luminosity Telescope (PLT) which is one of the dedicated luminometers for the Compact Muon Solenoid (CMS) experiment on the LHC at CERN.

 
%An accurate measurement of observed luminosity is one of the important parameter in particle physics experiments. This is because the occurrence of any rare physics events is directly proportional to the amount of collision events $\mathcal{L}$ measured by the detector. As such, luminosity parameter eventually sets limit on the confidence of our physics findings which is why it is equally important to know the backgrounds and uncertainties with our measurement. 

%This thesis describes the measurement of the luminosity delivered to the CMS's PLT detector at the LHC in pp collisions at a centre-of-mass energy of $\sqrt{14}= 14 TeV$ during 2015-2016 period. This thesis is structured the following way. Chapter \ref{ch:physBKGD} provides a brief overview of the Standard Model of particle physics and the concept of Luminosity, $mathcal{L}$. Chapter\ref{ch:expSetup} goes into detail about the experimental setup for the Pixel Luminosity Detector, and the VdM scan method used for data calibration. 

%PLT came online only in 2015; some systems for analyzing the machine conditions had to be built. Chapter \ref{ch:operations} provides the operational work done as part of the thesis. Finally, Chapter \ref{ch:reconstruction} and \ref{ch:Correction} deal with the track reconstruction, likelihood fits, and correction. Parameters taken from maximum likelyhood fits to the low-luminosity vdm scan are used as a reference to provide corrections to the luminosity at much higher intensity.

