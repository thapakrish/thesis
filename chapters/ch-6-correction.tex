\chapter{Determination of the Luminosity Correction} \label{ch:Correction}

\section{Introduction} \label{sec:corrIntro}
The goal of the luminosity measurement experiment is to find the true measure of proton-proton collisions. The LHC sends billions of protons on a head-head collision in a collection of protons called bunches, among which only some protons collide at a given time to produce secondary particles. PLT, located at about 171 cm away from the interaction point and at rapidity, $\eta$,  of $\sim$ 4, inclusively measures the charged particles. 

Within each filled bunch, the profile of the transverse density of protons is expected to be gaussian. Some protons, however, leak into neighboring bunches as seen in Figure ~\ref{fig:pBXfor}. Furthermore, protons can collide with elements within the beam pipe to produce spurious tracks. Some protons leave the ideal orbit and interact with rest gas atoms as the vacuum is not perfect, which causes secondary particle production resulting in extra tracks. 
Fig. \ref{fig:collcat} shows a schematic of tracks from several sources that can be distinguished via the track parameters--slopes, residuals. Generally, the tracks that PLT detects can be categorized as follows:


\begin{itemize}
    \item [1.] Tracks from IP   {\hfill + lumi}
    \item [2.] Tracks from IP with scatter  {\hfill + lumi}
    \item [3.] Tracks parallel to beam from collision with beam gas and obstructions far away from the IP    {\hfill - extra}
\end{itemize}


